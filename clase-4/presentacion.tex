\documentclass{beamer}
\usepackage[utf8]{inputenc}
\usepackage{verbatim}
\usepackage{minted}
\usepackage[spanish]{babel}
\usepackage{url}
\usetheme{Frankfurt}
\usecolortheme{seahorse}
\useoutertheme{shadow}
\useinnertheme{circles}
\graphicspath{ {./figures/} }
\title[Laboratorio]{Herramientas de trabajo (3)}
\author[Miguel]{Miguel Angel Piña Avelino}
\institute[UNAM]{
  Ingeniería de Software,\\
  Facultad de Ciencias, UNAM
}
\AtBeginSection[]{
  \begin{frame}
  \vfill
  \centering
  \begin{beamercolorbox}[sep=8pt,center,shadow=true,rounded=true]{title}
    \usebeamerfont{title}\insertsectionhead\par%
  \end{beamercolorbox}
  \vfill
  \end{frame}
}


\date{\today}
\begin{document}

\frame{\titlepage}
\begin{frame}
  \frametitle{Índice}
  \tableofcontents
\end{frame}

\section{Usando Git}

\begin{frame}[fragile]
  \frametitle{Agregando repositorios remotos}
  Hay que crear un repositorio remoto (en \textit{github}) y agregamos la
  siguiente instrucción:

\begin{minted}[fontsize=\scriptsize]{sh}
$ git remote add origin https://github.com/miguelpinia/something.git
\end{minted}

Hay que remplazar la url por la del repositorio. Una vez agregado el repositorio
remoto, hay que empujar todos los cambios que hemos hecho en nuestro repositorio
local:

\begin{minted}[fontsize=\scriptsize]{sh}
git push -u origin master
\end{minted}

\end{frame}

\begin{frame}[fragile]
  \frametitle{Obteniendo cambios desde repositorio}

  Supongamos que alguien hizo cambios en el repositorio y los empujó al
repositorio, para obtener esos cambios, ejecutamos la instrucción:

\begin{minted}[fontsize=\scriptsize]{sh}
$ git pull
\end{minted}

Para ver cuales fueron los cambios que se hicieron entre el último commit que
teníamos y el que obtuvimos ejecutamos:

\begin{minted}[fontsize=\scriptsize]{sh}
$ git diff HEAD
\end{minted}

\end{frame}

\begin{frame}[fragile]
  \frametitle{Creando ramas}
  Para agregar cambios sin modificar el contenido de la rama principal
  (rama master), vamos a crear una nueva rama donde podamos
  trabajar. Para hacer esto, realizamos lo siguiente:

\begin{minted}[fontsize=\scriptsize]{sh}
$ git checkout -b mi_nueva_rama
\end{minted}

A partir de aquí podemos realizar nuevos cambios. Para regresar o
moverse entre ramas, usando el comando checkout sin banderas.

\begin{minted}[fontsize=\scriptsize]{sh}
$ git checkout master
\end{minted}

\end{frame}

\begin{frame}[fragile]
  \frametitle{Mezclando ramas}
  Después de realizar una serie de cambios, es necesario que se mezclen con la
rama principal. Una forma limpia de hacerlo, es haciendo un rebase de la rama
con la que estamos trabajando respecto a la que vamos a mezclar.
\end{frame}

\begin{frame}[fragile]
  \frametitle{Mezclando ramas}
  La acción de rebase, se refiere a desplazar todos los cambios que
  existan en la rama a mezclar y que no se han integrado con la rama que
  estamos trabajando.

\begin{minted}[fontsize=\scriptsize]{sh}
$ git checkout mi_nueva_rama
$ git rebase master
$ git checkout master
$ git merge --no-ff mi_nueva_rama
$ git push
\end{minted}

\end{frame}

\begin{frame}[fragile]
  \frametitle{Eliminando ramas}
  La forma sencilla de eliminar ramas dentro de git es la siguiente:

\begin{minted}[fontsize=\scriptsize]{sh}
$ git branch -d mi_nueva_rama

\end{minted}


\end{frame}

\section{Java Server Faces}

\begin{frame}
  \frametitle{Java Server Faces}
  JSF es un framework MVC (Modelo-Vista-Controlador) basado en el API
  de Servlets que proporciona un conjunto de componentes en forma de
  etiquetas definidas en páginas XHTML mediante el framework Facelets.
  Antes de la especificación actual se utilizaba JSP para componer las
  páginas JSF.
\end{frame}


\begin{frame}[fragile]
  \frametitle{Java Server Faces}

  Entrando un poco más en detalle, JSF proporciona las siguientes
  características destacables:

  \begin{itemize}[<+->]
    \item Definición de las interfaces de usuario mediante vistas que
      agrupan componentes gráficos.
    \item Conexión de los componentes gráficos con los datos de la
      aplicación mediante los denominados beans gestionados.
    \item Conversión de datos y validación automática de la entrada del usuario.
    \item Navegación entre vistas.
    \item Internacionalización
    \item A partir de la especificación 2.0 un modelo estándar de
      comunicación Ajax entre la vista y el servidor.
  \end{itemize}
\end{frame}

\begin{frame}[fragile]
  \frametitle{Otras características}
  Otras características que tiene JSF son:

  \begin{itemize}[<+->]
    \item Soporte para Ajax
    \item Componentes múltiples
    \item Integración con Facelets
    \item Gestión de recursos (hojas de estilo, imágenes, etc.)
    \item Facilidad de desarrollo y despliegue
  \end{itemize}

\end{frame}

\begin{frame}[fragile]
  \frametitle{Java Server Faces}

  \begin{block}{Como se ejecuta}
    Tal y como hemos comentado, JSF se ejecuta sobre la tecnología de
    Servlets y no requiere ningún servicio adicional, por lo que para
    ejecutar aplicaciones JSF sólo necesitamos un contenedor de servlets
    tipo Tomcat o Jetty.
  \end{block}

\end{frame}

\begin{frame}[fragile]
  \frametitle{Java Server Faces}

  \begin{block}{Implementaciones}
    JSF es una especificación y, como tal, existen distintas
    implementaciones. Sun siempre proporciona una implementación de
    referencia de las tecnologías Java, que incluye en el servidor de
    aplicaciones GlassFish. En el caso de JSF, la implementación de
    referencia \emph{}s las dos implementaciones más usadas son:

    \begin{itemize}
    \item Mojarra
    \item MyFaces
    \end{itemize}
  \end{block}

\end{frame}

\begin{frame}
  \frametitle{Primefaces}

   PrimeFaces es una librería de componentes para JavaServer Faces
   (JSF) de código abierto que cuenta con un conjunto de componentes
   enriquecidos que facilitan la creación de las aplicaciones
   web. Primefaces está bajo la licencia de Apache License V2. Una de las
   ventajas de utilizar Primefaces, es que permite la integración con
   otros componentes como por ejemplo RichFaces.
\end{frame}

\section{Integrando JSF}

\begin{frame}
  \frametitle{Integrando JSF en Maven}
  Para ejemplificar la integración de JSF y la facilidad con la que se
  puede operar con JSF, vamos a implementar un formulario que permita
  simular el registro de un usuario.
\end{frame}


\begin{frame}[fragile]
  \frametitle{Integrando JSF en Maven}
  Lo primero que hay que hacer es tomar el ejemplo de Maven que se
  construyó en sesiones pasadas y modificar el archivo \texttt{pom.xml}
  en la sección de dependencias agregando el siguiente código para
  integrar las dependencias de JSF a nuestro proyecto:

  \begin{minted}[fontsize=\scriptsize]{xml}
  <!-- Agrega mos soporte para java server faces -->
  <dependency>
    <groupId>javax.servlet</groupId>
    <artifactId>jstl</artifactId>
    <version>1.2</version>
  </dependency>
  <dependency>
    <groupId>com.sun.faces</groupId>
    <artifactId>jsf-api</artifactId>
    <version>2.2.8</version>
    <type>jar</type>
  </dependency>
  \end{minted}
\end{frame}

\begin{frame}[fragile]
  \frametitle{Integrando JSF en Maven}
  \begin{minted}[fontsize=\scriptsize]{xml}
  <dependency>
    <groupId>com.sun.faces</groupId>
    <artifactId>jsf-impl</artifactId>
    <version>2.2.8-19</version>
  </dependency>
  <!-- Soporte para primefaces -->
  <dependency>
    <groupId>org.primefaces</groupId>
    <artifactId>primefaces</artifactId>
    <version>6.0</version>
  </dependency>
  \end{minted}
\end{frame}

\begin{frame}[fragile]
  \frametitle{Integrando JSF con Maven}
  El siguiente archivo a modificar es web.xml de modo que luzca como
  el siguiente código

  \begin{minted}[fontsize=\scriptsize]{xml}
  <?xml version="1.0" encoding="UTF-8"?>
  <web-app version="3.1"
           xmlns="http://xmlns.jcp.org/xml/ns/javaee"
           xmlns:xsi="http://www.w3.org/2001/XMLSchema-instance"
           xsi:schemaLocation="http://xmlns.jcp.org/xml/ns/javaee
               http://xmlns.jcp.org/xml/ns/javaee/web-app_3_1.xsd">
    <display-name>Archetype Created Web Application</display-name>

    <context-param>
      <param-name>javax.faces.PROJECT_STAGE</param-name>
      <param-value>Development</param-value>
    </context-param>
    <servlet>
      <servlet-name>Faces Servlet</servlet-name>
      <servlet-class>javax.faces.webapp.FacesServlet</servlet-class>
      <load-on-startup>1</load-on-startup>
    </servlet>
  \end{minted}
\end{frame}

\begin{frame}[fragile]
  \frametitle{Integrando JSF con Maven}
  \begin{minted}[fontsize=\scriptsize]{xml}
    <servlet-mapping>
      <servlet-name>Faces Servlet</servlet-name>
      <url-pattern>/faces/*</url-pattern>
    </servlet-mapping>
      <servlet-mapping>
      <servlet-name>Faces Servlet</servlet-name>
      <url-pattern>*.xhtml</url-pattern>
    </servlet-mapping>
    <servlet>
      <servlet-name>servlet</servlet-name>
      <servlet-class>com.miguel.proyecto.MiServlet</servlet-class>
    </servlet>
    <servlet-mapping>
      <servlet-name>servlet</servlet-name>
      <url-pattern>/foo</url-pattern>
    </servlet-mapping>
  </web-app>
  \end{minted}
\end{frame}

\begin{frame}[fragile]
  \frametitle{Integrando JSF con Maven}
  Ahora que ya tenemos el soporte de las bibliotecas de JSF dentro del
  proyecto, vamos a comenzar a crear la página para validar la
  creación de inicio de sesión. Para esto comenzaremos con el concepto
  de \texttt{Managed Bean}.
\end{frame}

\begin{frame}[fragile]
  \frametitle{Integrando JSF con Maven}
    \begin{block}{¿Qué es un Managed Bean?}
    Un \texttt{Managed Bean} es una clase que sigue la nomenclatura de
    los Java Beans. El objetivo de estos objetos es controlar el estado de
    la página web. \texttt{JSF} va a administrar automáticamente los
    managed beans.
  \end{block}
\end{frame}

\begin{frame}[fragile]
  \frametitle{Integrando JSF con Maven}
  Ahora vamos a integrar un bean que represente un usuario del
  sistema.

  \begin{minted}[fontsize=\scriptsize]{java}
  package com.miguel.proyecto.web;

  public class Usuario {

      private String usuario;
      private String contraseña;
      private String confirmacionContraseña;

      public String getUsuario() {
          return usuario;
      }

      public void setUsuario(String usuario) {
          this.usuario = usuario;
      }

  \end{minted}
\end{frame}

\begin{frame}[fragile]
  \frametitle{Integrando JSF con Maven}
  \begin{minted}[fontsize=\scriptsize]{java}
      public String getContraseña() {
          return contraseña;
      }

      public void setContraseña(String contraseña) {
          this.contraseña = contraseña;
      }

      public String getConfirmacionContraseña() {
          return confirmacionContraseña;
      }

      public void setConfirmacionContraseña(String confirmacionContraseña) {
          this.confirmacionContraseña = confirmacionContraseña;
      }

  }
  \end{minted}
\end{frame}

\begin{frame}[fragile]
  \frametitle{Integrando JSF con Maven}

  Agregamos un bean manejado de JSF.

  \begin{minted}[fontsize=\scriptsize]{java}
  package com.miguel.proyecto.web;

  import java.util.Locale;
  import javax.faces.application.FacesMessage;
  import javax.faces.bean.ManagedBean;
  import javax.faces.bean.RequestScoped;
  import javax.faces.context.FacesContext;

  @ManagedBean
  @RequestScoped
  public class RegisterController {

  private Usuario user = new Usuario();

      public Usuario getUser() {
          return user;
      }

      public void setUser(Usuario user) {
          this.user = user;
      }

  \end{minted}
\end{frame}

\begin{frame}[fragile]
  \frametitle{Integrando JSF con Maven}
  \begin{minted}[fontsize=\scriptsize]{java}
      public RegisterController() {
          FacesContext.getCurrentInstance()
                      .getViewRoot()
                      .setLocale(new Locale("es-Mx"));
      }

      public String addUser() {
          if (!user.getContraseña()
                   .equals(user.getConfirmacionContraseña())) {
              FacesContext.getCurrentInstance()
                          .addMessage(null
                          , new FacesMessage(FacesMessage.SEVERITY_ERROR
                , "Fallo de registro: Las contraseñas deben coincidir", ""));
          } else {
              FacesContext.getCurrentInstance()
                          .addMessage(null,
                           , new FacesMessage(FacesMessage.SEVERITY_INFO,
              "Felicidades, el registro se ha realizado correctamente", ""));
              user = null;
          }
          return null;
      }

  }
  \end{minted}
\end{frame}

\begin{frame}[fragile]
  \frametitle{Integrando JSF con Maven}

  Y finalmente la página que queremos visualizar. En este caso un
  ejemplo sencillo de validación de contraseñas.

  \begin{minted}[fontsize=\scriptsize]{html}
    <?xml version='1.0' encoding='UTF-8' ?>
    <!DOCTYPE html PUBLIC "-//W3C//DTD XHTML 1.0 Transitional//EN"
      "http://www.w3.org/TR/xhtml1/DTD/xhtml1-transitional.dtd">
    <html xmlns="http://www.w3.org/1999/xhtml"
          xmlns:h="http://xmlns.jcp.org/jsf/html"
          xmlns:p="http://primefaces.org/ui">
      <h:head>
        <title>Ejemplo de JavaServer Faces</title>
       </h:head>
    <h:body>

      <h1>Formulario de registro</h1> <br/>
  \end{minted}
\end{frame}

\begin{frame}[fragile]
  \frametitle{Integrando JSF con Maven}
  \begin{minted}[fontsize=\scriptsize]{html}
      <h:form id="myForm">
        <table>
          <tr>
            <td>Nombre de usuario</td>
            <td><p:inputText
                 value="#{registerController.user.usuario}"
                 required="true" id="Username" size="10"/></td>
          </tr>
          <tr>
            <td>Contraseña</td>
            <td><p:password
                 value="#{registerController.user.contraseña}"
                 required="true" feedback="true" id="Password"/></td>
          </tr>
          <tr>
            <td>Confirmar Contraseña</td>
            <td><p:password
                 value="#{registerController.user. confirmacionContraseña}"
                required="true" feedback="true" id="ConfirmPassword"/></td>
          </tr>
  \end{minted}
\end{frame}

\begin{frame}[fragile]
  \frametitle{Integrando JSF con Maven}
  \begin{minted}[fontsize=\scriptsize]{html}
          <tr>
            <td colspan="2" align="center">
              <p:commandButton action="#{registerController.addUser}"
                               value="Registrar"/>
              <p:commandButton value="Reset" update="myForm"
                               process="@this">
                <p:resetInput target="myForm" />
              </p:commandButton>
            </td>
          </tr>
        </table>
        <table>
          <tr>
            <td>
              <p:messages id="messages" autoUpdate="true"
                          closable="true" />
            </td>
          </tr>
        </table>
      </h:form>
    </h:body>
  </html>
  \end{minted}
\end{frame}

\begin{frame}[fragile]
  \frametitle{Integrando JSF con Maven}
  Y terminamos levantando nuestro sitio con:

  \begin{minted}[fontsize=\scriptsize]{sh}
    mvn tomcat7:run
  \end{minted}

  E ingresamos a la siguiente url.\\

  \url{http://localhost:8080/mi-primer-aplicacion-web/registro.xhtml}
\end{frame}
\end{document}
