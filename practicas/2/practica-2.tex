\documentclass[11pt]{article}
\usepackage[utf8]{inputenc}
\usepackage[spanish]{babel}
\usepackage{amsmath}
\usepackage{amsfonts}
\usepackage{amssymb}
\usepackage{makeidx}
\usepackage{graphicx}
\usepackage{verbatim}
\usepackage[left=2cm,right=2cm,top=2cm,bottom=2cm]{geometry}
\graphicspath{ {./figures/} }
\author{Miguel Angel Piña Avelino}
\date{\today}
\title{Práctica 2}
\setlength\parindent{0pt}
\begin{document}

\maketitle

\section*{Práctica de laboratorio}

Uno de los problemas que suele suceder cuando se desarrolla código de
forma colaborativa, es el estilo de código que cada uno de los
integrantes del equipo aplica al código que desarrolla.\\
\newline
Sin embargo, se busca que el código de todos sea uniforme respecto a
algún estándar como el que provee Java para su código fuente.\\
\newline
La ventaja, es que actualmente existen herramientas que facilitan
esto e incluso los editores de texto y algunos IDEs como Netbeans
ayudan en estas tareas.

\section*{Cosas por hacer}

Buscar e integrar con su proyecto la herramienta Checkstyle, la cuál
ayudará a identificar si el código que desarrollan se encuentra bajo
escrito siguiendo los estándares de codificación.\\
\newline
Una vez integrado, entregar un reporte por equipo indicando cuales son
los casos que se validan al momento de ejecutar la tarea \textbf{mvn
checkstyle:checkstyle} y como es resolvieron la integración de la
herramienta con su proyecto.


\subsection*{Entregables}

El reporte puede ser enviado en formato PDF o txt, indicando el nombre
de los integrantes y la url de su repositorio, a la dirección
\textit{miguel\_pinia@ciencias.unam.mx} a más tardar el día \textbf{16 de mayo
a las 23:59}.

\end{document}
%%% Local Variables:
%%% mode: latex
%%% TeX-master: t
%%% End:
