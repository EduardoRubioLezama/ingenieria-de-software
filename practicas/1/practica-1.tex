\documentclass[11pt]{article}
\usepackage[utf8]{inputenc}
\usepackage[spanish]{babel}
\usepackage{amsmath}
\usepackage{amsfonts}
\usepackage{amssymb}
\usepackage{makeidx}
\usepackage{graphicx}
\usepackage{verbatim}
\usepackage[left=2cm,right=2cm,top=2cm,bottom=2cm]{geometry}
\graphicspath{ {./figures/} }
\author{Miguel Angel Piña Avelino}
\date{\today}
\title{Práctica 1}
\setlength\parindent{0pt}
\begin{document}
\maketitle

\section*{Práctica de laboratorio}

En clase de laboratorio comenzamos a trabajar con git, postgresql y
netbeans. En esta práctica vamos a comenzar a trabajar en tres
etapas. La primera es trabajar con Git, la segunda es con MySQL y la
tercera es con Netbeans + Maven.

\subsection*{Git}
Cada equipo deberá de tener un repositorio de código basado en git. En
clase vimos algunos ejemplos de repositorios web, entre los que
destacaban \textit{Github} y \textit{Bitbucket}. Dependiendo de la
elección del equipo, deberán tener su repositorio con acceso  para
todos los integrantes y añadirme con permisos de escritura.\\

Una vez realizado esto, cada integrante del equipo deberá de hacer lo siguiente:
\begin{itemize}
  \item Crear una rama nombrada siguiendo el formato
    {PrimerNombre}-{Apellido}-{Practica1}
  \item Crear una carpeta nombrada SQL
\end{itemize}

\subsection*{Base de datos}

Una vez realizado lo anterior, cada integrante del equipo sobre su
rama de desarrollo deberá crear un archivo SQL llamado
\textit{proyecto-servicio-social.sql} dentro de la carpeta SQL.\\

Este archivo debe de contener una primera definición de las relaciones
que soportarán los casos de uso que cada integrante tiene asignados,
es decir, las relaciones que necesiten para resolver sus casos de
uso.\\

\subsection*{Netbeans}

Después de finalizar lo anterior, utilizando Netbeans, el responsable
técnico debe de crear un proyecto web con el nombre que el equipo
decida (con Maven), dentro de la rama \textit{master} de git y
publicarla para que todos los demás puedan mezclarla con sus ramas
personales.

Cada integrante deberá de mezclar los cambios realizados en la rama
\textit{master} con su rama personal y una vez que cada uno tenga su
copia del proyecto, dentro de este deberán hacer lo siguiente:

\begin{itemize}
  \item Cada integrante deberá de hacer los prototipos de interfaz que
    le tocaron.
  \item Estos prototipos deberán de ser realizados mediante archivos
    \textit{.xhtml ó html} y dentro de ellos usar \textbf{sólo} HTML y CSS. Si
    es el caso.
  \item Los archivos HTML (XHTML) deben estar dentro de la carpeta web del
    proyecto.
\end{itemize}

\subsection*{Mezclando el trabajo}

Una vez que todos hayan terminado de realizar lo anterior, todos deben
de empujar sus ramas para ser rastreadas por el repositorio de código,
el responsable técnico deberá de mezclar todas las ramas y corregir
los conflictos que se den al mezclar las ramas.\\
Para que esto pueda ser realizado eficientemente, todos los
integrantes del equipo deben de mantener comunicación constante y
discutir cuales cambios son los que se quedan en caso de conflictos.

\subsection*{Fecha de revisión}
El día en que se revisará el trabajo que realizaron será el día 13 de
marzo del 2018. Previamente deberán entregar un reporte por equipo
mostrando paso a paso las cosas que tuvieron que hacer, envíandolo al
correo \texttt{miguel\_pinia@ciencias.unam.mx}.

\end{document}
%%% Local Variables:
%%% mode: latex
%%% TeX-master: t
%%% End:
